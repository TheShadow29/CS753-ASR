\documentclass{article}
\usepackage[a4paper, tmargin=1in, bmargin=1in]{geometry}
\usepackage[utf8]{inputenc}
\usepackage{graphicx}
\usepackage[justification=centering]{caption}

% \usepackage{parskip}
\usepackage{pdflscape}
\usepackage{listings}
\usepackage{hyperref}
\usepackage{caption}
\usepackage{subcaption}
\usepackage{float}
\usepackage{amsmath,amssymb}

\DeclareMathOperator{\E}{\mathbb{E}}


\title{CS 753 : Automatic Speech Recognition Assignment 1 Problem 1}
\author{Arka Sadhu - 140070011}
\date{\today}

\begin{document}
\maketitle
\section{Part A}
We are given that  the probability of sending bit $0$ is $p$ and sending bit $1$ is $1-p$. It is also mentioned that the bits are independent and indentically distributed.\\
Also it is given that the channel is noisy and therefore a bit can be flipped with a probability of $\delta$, and therefore  remains unchanged with probability $1-\delta$.\\
We need to find the probability that the $k^{th}$ symbol received is $0$.\\

Since the bits are independent the probability of $k^{th}$ symbol is not dependent upon the other symbols. Also it can be assumed that the channel doesn't have a time delaying effect and therefore it affects only the considered bit with a probability given by $\delta$.

The $k^{th}$ symbol can be received as $0$ in two ways:
\begin{itemize}
\item The sent bit was $0$ and no flipping occurred. The probability of this happening is $P1 = p(1-\delta)$.
\item The sent bit was $1$ and flipping occurred. The probability of this is $P2 = (1-p)\delta$.
\end{itemize}

Since the two events are mutually exclusive, therefore the probabilities simply add up.
Therefore
\begin{equation}
  \label{eq:q1a}
  Pr(\mathrm{k^{th} symbol received = 0}) = p(1-\delta) + (1-p)\delta
\end{equation}

\section{Part B}
A repetition code is used such that $0$ is encoded as $000$ and $1$ as $111$ and the decoder outputs a majority bit on the receiving 3-bit string.\\
We need to find the probability of correctly decoding $0$.

It is easy to see that $0$ will be correctly decoded when at most 1 bit is flipped. This can happen in one of the following ways:
\begin{itemize}
\item None of the bits get flipped. $Pr(\mathrm{No bit flipped}) = (1-\delta)^3$
\item Exactly one of the bits gets flipped. $Pr(\mathrm{Exactly one bit flipped}) = 3 \delta (1 - \delta)^2$
\end{itemize}
Again the above are mutually exclusive, the probabilities add up.
Therefore
\begin{equation}
  \label{eq:q1b}
  Pr(\mathrm{0 being correctly decoded}) = (1 - \delta)^3 + 3 \delta (1 - \delta)^2
\end{equation}

\section{Part C}
Now it is said that each bit is transmitted unencoded. But the decoder is allowed to output $\{0,1,\bot\}$. \\
If the decoder output is incorrect then the penalty points are $2$ and if it outputs it to be $\bot$ then it gets penalty $1$.\\
The aim is to find the best decoding strategy which can minimize the expected penalty. We are given $\delta=\frac{1}{8}$ and want the best strategy for $p \in [0,1]$.\\

We first do the problem more generally and then impose the conditions given to us. We take that on output of $\bot$ the penalty is $1$ and the output on incorrect bit the penalty is $X_1$. Also we don't put the value of $\delta$ till the last step.\\

The decoder either sees $0$ or $1$ and chooses on of the three options $\{0,1,\bot\}$. We assign probabilities to each of the six cases.
\begin{center}
  \begin{tabular}[H]{|l|l|l|l|}
    \hline
    & 0 & 1 & $\bot$\\
    \hline
    0 & $\rho_1$ & $\rho_2$ & $\rho_3$\\
    \hline
    1 & $\rho_4$ & $\rho_5$ & $\rho_6$\\
    \hline
  \end{tabular}
\end{center}

Let the expected penalty be denoted by $\E[Pen]$. Clearly there are four ways in which penalty can arise:
\begin{itemize}
\item Sender sends $0$, bit is not flipped (stays $0$). Penalty arises when detector declares it either $1$ or $\bot$. $$\E[Pen|case_1] = X_1\rho_2 + \rho_3$$
  $$Pr(case_1) = p(1-\delta)$$
\item Sender sends $0$, bit is flipped (becomes $1$). Penalty arises when detector declares it either $1$ or $\bot$. $$\E[Pen|case_2] = X_1\rho_5  + \rho_6$$
  $$Pr(case_2) = p\delta$$
\item Sender sends $1$, bit is flipped (becomes $0$). Penalty arises when detector declares it either $0$ or $\bot$. $$\E[Pen|case_3] = X_1 \rho_1 + \rho_3$$
  $$Pr(case_3) = (1-p)\delta$$
\item Sender sends $1$ bit is not flipped (stays $1$). Penalty arises when detector declares it either $0$ or $\bot$. $$\E[Pen|case_4] = X_1 \rho_4 + \rho_6$$
  $$Pr(case_4) = (1-p)(1-\delta)$$
\end{itemize}

Also $$\E[Pen] = \sum_i\E[Pen|case_i]Pr(case_i)$$
\begin{equation}
  \label{eq:c1}
  \E[Pen] = \rho_1 (X_1 \delta (1-p)) + \rho_2 (X_1 p (1 - \delta)) + \rho_3(p(1-\delta) + \delta(1-p)) + \rho_4(X_1(1-p)(1-\delta)) + \rho_5(X_1p\delta) + \rho_6(p\delta + (1-p)(1-\delta)) %
\end{equation}

% $$\E[Pen] = \rho_1 (X_1 \delta (1-p)) + \rho_2 (X_1 p (1 - \delta)) + \rho_3(p(1-\delta) + \delta(1-p)) + \rho_4(X_1(1-p)(1-\delta)) + \rho_5(X_1p\delta) + \rho_6(p\delta + (1-p)(1-\delta))$$ %
\vspace{0.5cm}
Now we want to minimize the objective function in \ref{eq:c1}. The parameters are $\{\rho_1, \rho_2, \rho_3, \rho_4, \rho_5, \rho_6\}$. Moreover the parameters are constrained in the following ways:
\begin{equation}
  \label{eq:c2}
  \begin{aligned}
  \rho_1 + \rho_2 + \rho_3 &= 1\\
  \rho_4 + \rho_5 + \rho_6 &= 1
  \end{aligned}
\end{equation}

\begin{equation}
  \label{eq:c3}
  \begin{aligned}
    \rho_1 \ge 0 ,& \hspace{0.3cm}\rho_2 \ge 0 ,& \rho_3 \ge 0\\
    \rho_4 \ge 0 ,& \hspace{0.3cm}\rho_5 \ge 0 ,& \rho_6 \ge 0
  \end{aligned}
\end{equation}

We note that the inequality of $ \le 1$ is implied when the two constraints \ref{eq:c2} and \ref{eq:c3} are met. It is quite easy to see that the above can be posed as a linear programming problem with \ref{eq:c2} and \ref{eq:c3} being the constraints $Ax=b$ and $x \ge 0$. Therefore we can be guarenteed the solution to the above minimization problem lies in on the corners of the vector space $\{\rho_i\}_{i=1}^6$.

Also from the simplex algorithm it can be directly seen that the solutions will be of the form such that one of $\rho_i|_{i=1}^3$ will be $1$ and other two $0$, and the same for the $\rho_i|_{i=4}^6$. Since our objective is to minimize we can simply chooose the $\rho_i$ whose coefficient is minimum. Therefore calculating the coefficients at once gives us the decoding strategy at once. Also since the probabilities come out to be $0$ or $1$, the decoding strategy turns out to be deterministic.

Now we simply get the coefficient values for $X_1 = 2$ and $\delta = \frac{1}{8}$.
\begin{table}[H]
\centering
\caption{My caption}
\label{my-label}
\begin{tabular}{|l|l|l|l|l|l|l|}
\hline
          & $\rho_1$        & $\rho_2$       & $\rho_3$           & $\rho_4$           & $\rho_5$      & $\rho_6$           \\ \hline
$coeff_i$ & $\frac{1-p}{4}$ & $\frac{7p}{4}$ & $\frac{6p + 1}{8}$ & $\frac{7(1-p)}{4}$ & $\frac{p}{4}$ & $\frac{7 - 6p}{8}$ \\ \hline
\end{tabular}
\end{table}
For $\rho_1, \rho_2, \rho_3$ we note that when
\begin{itemize}
\item When $p \in [0,\frac{1}{8}]$, $coeff_2 \le coeff_1$ and $coeff_2 \le coeff_3$. Thus we choose $\rho_2=1$ and $\rho_1 = \rho_3 = 0$.
\item When $p \in [\frac{1}{8}, 1]$, $coeff_1 \le coeff_2$ and $coeff_1 \le coeff_3$. Thus we choose $\rho_1 = 1$ and $\rho_2 = \rho_3 = 0$.
\end{itemize}
It is interesting to note that $\rho_3$ is always 0.

For $\rho_4$, $\rho_5$, $\rho_6$ we note that when
\begin{itemize}
\item When $p \in [0, \frac{7}{8}]$, $coeff_5 \le coeff_4$ and $coeff_5 \le coeff_6$. Thus we choose $\rho_5 = 1$ and $\rho_4 = \rho_6 = 0$.
\item When $p \in [\frac{7}{8}, 1]$, $coeff_4 \le coeff_5$ and $coeff_4 \le coeff_6$. Thus we choose $\rho_4 = 1$ and $\rho_5 = \rho_6 = 0$.
\end{itemize}
Again $\rho_6$ is always 0.

This means that the decoder never chooses the $\bot$ symbol.

Further we can get the decoding strategy for each of $p \in [0,1]$.
\begin{itemize}
\item For $p \in [0, \frac{1}{8}]$
\end{itemize}

\end{document}
